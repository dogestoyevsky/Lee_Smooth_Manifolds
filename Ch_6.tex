\documentclass[10pt,letter]{article}
\usepackage{amsmath,amssymb,breqn,enumitem,fullpage,graphicx,setspace,mathtools,stmaryrd}
\onehalfspacing
\usepackage{fullpage}

\begin{document}
\noindent Github/dogestoyevsky \\
June 2017
\begin{center}
\textbf{Chapter Six: Sard's Theorem}\\ Lee, \textit{An Introduction to Smooth Manifolds}

\line(1,0){250}
\end{center}

\subparagraph{6.2} {\bf Immersion of an $n$-dimensional manifold into $\mathbb{R}^{2n}$.} We assume that $M$ is an embedded submanifold of $\mathbb{R}^{2n+1}$. We saw in Problem 5.6 that $UM$ is an $2n-1$ dimensional submanifold of $T\mathbb{R}^{2n+1}$. Since $\mathbb{R}P^{2n}$ has dimension $2n$, Corollary 6.11 to Sard's theorem implies that the image of the $UM$ unde $G$ has measure $0$. Since $U_{2n+1} = \lbrace [v_0,...,v_{2n}] \in \mathbb{R}P^{2n}: v_{2n} \neq 0 \rbrace$ is open, this means that there exists $[w] \in U_{2n+1}$ not contained in $G(UM)$. Then $w$ is a vector in $\mathbb{R}^{2n+1}$ not contained in $\mathbb{R}^{2n}$ such that $(p,w) \not \in T_pM$ for any $p \in M$. 

Let $F: \mathbb{R}^{2n+1} \rightarrow \mathbb{R}^{2n}$ be the quotient map by $w$. Since $F$ is linear, it is smooth and equal to its own differential under the canonical identification. We want to show that $DF_p$ is injective for $p \in M$. Say otherwise that $DF_p(v) = 0$ for $v \in T_pM$. That implies that $v =  \lambda w$, which implies that $(p,w) \in T_pM$, a contradiction. Therefore $F$ is an immersion on $M$. 

I don't see exactly where this fails if the boundary isn't empty. Does Problem 5.6 still hold? Otherwise, perhaps to issue is that the negative of $w$ might be orthogonal to $DF$ but not contained in $UM$. 

\subparagraph{6.3} {\bf Smooth lower approximation of a continuous function that vanishes on a closed set.} Let $f$ be a nonnegative smooth function such that $f^{-1}(0) = A$ (Theorem 2.29). Then $F = \frac{f}{f+1}$ is a function of the same type and $F < 1$. Let $e$ be a smooth function such that $0 < e(x) < \delta(x)$ (Corollary 6.22). Then $\tilde{\delta} = F \cdot e$ is the desired function. 

\subparagraph{6.4} {\bf Smooth approximations to continuous functions.} 
\begin{enumerate}[label=(\alph*)]
\item Take $\tilde{d}$ as an Problem 6.3. Let $\tilde{G}$ be a smooth $\tilde{d}$-close approximation to $F \vert_{M \setminus B}$ (Theorem 6.21) and define
\[
\tilde{F}(x) = \begin{cases}
    \tilde{G}(x)  & \text{if } x \in M \setminus B  \\        
    F(x) & \text{if } x \in B 
\end{cases}.
\]
Clearly $\tilde{F}$ is $\delta$-close to $F$. To see that $\tilde{F}$ is continuous, take $p \in \delta M$ and let $(U,\phi)$ be a smooth chart such that $p \in U$. Now on the one hand, by the continuity of $F$ there is a neighborhood $B_{\epsilon_1}(p)$ such that $\vert F(q) - F(p) \vert < \epsilon$ for $q \in B_{\epsilon_1}(p) \cap B$. On the other hand, since  $\tilde{d}(p)  = 0$, there is a neighborhood $B_{\epsilon_2}(p)$ such that $\vert \tilde{G}(q) - F(q) \vert < \epsilon$ for $q \in B_{\epsilon_2}(p) \cap M \setminus B$. Let $B = B_{\epsilon_1}(p) \cap B_{\epsilon_2}(p)$. It is clear that $\vert \tilde{F}(q) - \tilde{F}(p) \vert < \epsilon$ for $q \in B$. This shows that $\tilde{F}$ is continuous. 
\item Taking $\tilde{F}$ constructed here, the proof of Theorem 6.26 goes through unchanged, and clearly the homotopy is relative to $B$.
\end{enumerate}

\subparagraph{6.5} {\bf Tubular neighborhood of points with unique closest point.}  Let $M$ be an embedded submanifold of $\mathbb{R}^n$ and let $y \in \mathbb{R}^n \setminus M$. If $y \in N$ satisfies $d(y,x) = \min \limits_{x' \in N} \, d(y,x')$, then for any curve $\gamma: I \rightarrow M$ such that $\gamma(0) = x$, $F(t) = \Vert y - \gamma(t) \Vert^2$ must have a minimum at $t = 0$.  Differentiating, this implies that $y - \gamma(t)$ is orthogonal to $\gamma'(t)$. Since $\gamma$ was arbitrary, this implies that $y - \gamma(t)$ is orthogonal to $T_xM$. 

Therefore, if $x \in M$ is a closest point to $y$, the $x-y \in N_xM$. Now, take some $\delta$ such that $E$ gives a diffeomorphism between a neighborhood of $x$ containing $\overline{B_{\delta}}(x) \subset \mathbb{R}^n$. By the slice chart condition, we can assume that $\delta$ is small enough that $S = \overline{B_{\delta}}(x) \cap M$ is closed in $\mathbb{R}^n$.  Now consider $ B_{\delta/2}(x)$. By the compactness of $S$, every element of $B_{\delta/2}(x)$ has a (not necessarily unique) closest element in $S$. But a closest element of $S$ must be a closest element of $M$, since $y \in B_{\delta/2}(x)$ implies that $\vert y -x \vert < \delta/2$. 

Further, we claim that for every $y \in B_{\delta/2}(x)$, there is exactly one such closest element. $E$ restricts to a diffeomorphism between $E^{-1}(\overline{B_{\delta}}(x) \subset \mathbb{R}^n) \subseteq NM$ and $\overline{B_{\delta}}(x) \subset \mathbb{R}^n$. Now say there are two points $x_1$ and $x_2$ such that $y-x_1 \in N_{x_1}M$ and $y-x_2 \in N_{x_2}M$. Then since $\Vert y-x_i \Vert < \delta$ ($i = 1,2$), $(x_1,y-x_i) \in E^{-1}(\overline{B_{\delta}}(x) \subset \mathbb{R}^n)$ and $E(x_1,y-x_1) = E(x_2,y-x_2)$, a contradiction. Therefore the closest point is unique. This proves that for every $x \in M$, there is a radius $\rho_x$ so that all point in $B_{\rho_x}(x)$ have a unique closest point in $M$. 

Now, define $\rho': M \rightarrow \mathbb{R}_{> 0}$ to be the supremum of all such $\rho(x)$. An argument like the one in the text shows that $\rho'$ is continuous. Then $\lbrace (x,y): y \in N_xM, \Vert y \Vert < \rho'(x) \rbrace$ is the desired tubular neighborhood. It is clear from the above discussion of orthogonality that $r(y)$ gives the unique closest point in this case. 

\subparagraph{6.6} {\bf Points exactly $\epsilon$ away from an embedded submanifold.} Let $T \subset \mathbb{R}^n$ be the tubular neighborhood of $M$ satisfying the conditions of the previous problem. Since $M$ is compact, there is $\epsilon > 0$ such that if $d(y,M) < \epsilon$, $y \in T$. Let $N_{\epsilon}M = \lbrace (x,y) \in NM: y \in N_xM, \Vert y \Vert \leq \epsilon \rbrace$ and $\delta N_{\epsilon}M = \lbrace (x,y) \in NM: y \in N_xM, \Vert y \Vert = \epsilon \rbrace$. Now $F(x,v) = \Vert v \Vert$ is a smooth function on $NM$ such that $d$ is a regular value for all $d > 0$. Therefore, by Proposition 5.47, $\overline{N_{\epsilon}M}$ is a regular domain. By Proposition 5.46, its boundary is the topological boundary $\delta N_{\epsilon}M$, and by Theorem 5.11 this boundary is an embedded codimension-1 submanifold. Since $E$ restricts to a diffeormorphism on a neighborhood of $\overline{N_{\epsilon}M}$ and $\overline{M_{\epsilon}} = E(\overline{N_{\epsilon}M})$ and $\delta M_{\epsilon} = E(\delta N_{\epsilon}M)$, this completes the proof. 

\subparagraph{6.8} ? 

\subparagraph{6.10} {\bf Preimage of tangent space under transverse map.} By Theorem 6.30, $W$ is an embedded submanifold of $N$. Given $p \in W$ and $v \in T_pW$, let $\gamma: I \rightarrow W$ be a curve such that $\gamma(0) = p$ and $\gamma'(0) = v$. Then $F \circ \gamma$ is a smooth curve in $X$ such that $(F \circ \gamma)'(0) = DF_p v$. This proves that $D_p W \subseteq (DF_p)^{-1}(T_{F(p)}X)$. On the other hand, let $A$ be a complementary subspace to $T_pW$ in $T_pN$. Then by the definition of transversality, $T_{F(p)}N = T_{F(p)}X + dF_p(T_pW) + dF_p(A) = T_{F(p)}X + dF_p(A)$, so by dimension counting (Theorem 6.30), we must have that $dF_p(A) \cap   T_{F(p)}X  = 0$.

Now if $X$ and $X'$ are embedded submanifolds of $M$ that intersect transversally, then $i: X \hookrightarrow M$ is transverse to $X'$. But $X \cap X' = i^{-1}(X')$, so by the above $(di_p)^{-1}(T_p(X')) = T_p(X \cap X')$, and on the other hand since $T_pX = (di_p)^{-1}(X)$, $T_pX \cap T_pX' = (di_p)^{-1}(T_p(X'))$. (This is basically a question of notation -- the intersections are taken to mean intersections under the inclusion.) 

\subparagraph{6.12} {\bf Approximation $M \rightarrow \mathbb{R}^N$ by smooth immersions when $M$ is compact.} Let $n = \text{dim }M$. By Theorem 6.18, for $N \geq 2n$ there is an immersion $G: M \rightarrow \mathbb{R}^N$. Since $M$ is compact, $G(M)$ is bounded, so $\vert G(x) \vert < K$ for all $x \in M$ and some $K$. Now since $dG_p$ is injective for all $p \in M$, for all $p \in M$ and any $\epsilon' > 0$, there exists $0 < \epsilon < \epsilon'$ such that $d(F + \epsilon G)_p$ is injective. Let $H_{\epsilon} = F + \epsilon G$. Then $\vert H_{\epsilon}(x) - F(x) \vert \leq \epsilon K$. 

\subparagraph{6.14} {\bf Counterexample to Theorem 6.30 if transversality assumption removed.} Let $f$ be the smooth function $\mathbb{R}^{n-1} \rightarrow \mathbb{R}$ such that $f^{-1}(0) = A$, guaranteed by Theorem 2.29, and let $F: \mathbb{R}^{n} \rightarrow \mathbb{R}$ be the function $F(x,y) = f(x) + y$ where $x \in \mathbb{R}^{n-1}$ and $y \in \mathbb{R}$. Then $F$ is a smooth submersion, since $dF_p = (df_p,1)$, so $F^{-1}(0)$ is a proper embedded hypersurface (Corollary 5.13). The contradiction here is that $S \cap S' = A$, $S$ has codimension $1$, $S'$ has comdimension $1$, but $A$ can have any positive codimension.
\end{document}
