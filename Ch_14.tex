\documentclass[10pt,letter]{article}
\usepackage{amsmath,amssymb,breqn,enumitem,fullpage,graphicx,setspace,mathtools,pst-node,stmaryrd,tikz-cd}
\onehalfspacing
\usepackage{fullpage}

\begin{document}
\noindent Github/dogestoyevsky \\
July 2017
\begin{center}
\textbf{Chapter Fourteen: Differential Forms}\\ Lee, \textit{An Introduction to Smooth Manifolds}

\line(1,0){250}
\end{center}
\subparagraph*{14.1} {\bf Wedge product and linear independence.} It is clear from anticommutativity (Proposition 14.11c) that $\omega^1 \wedge ... \wedge \omega^n = 0$ if $\omega_i = \omega_j$ for $i \neq j$. Now say $\sum_{k=1}^n c_k \omega^k = 0$, and assume without loss of generality that $c_1 \neq 0$. Then $\omega^1 = \sum_{k=2}^n c_k' \omega^k$. Therefore 
\[ \omega^1 \wedge ... \wedge \omega^n = \sum_{k=2}^n c_k' \omega^k \wedge \omega^2  \wedge ... \wedge \omega^n = 0 \] since every term is zero. 

On the other hand, say $\omega^1,...,\omega^n$ are independent. Extend this collection to a basis $(\omega_k)$ for $V^{\ast}$ and then let $(e_k)$ be a dual basis for $E$, so that $\omega^i(e_j) = \delta_{ij}$. Then $(\omega^j(v_i))$ is the $n \times n$ identity matrix, so by Proposition 14.11, \[ \omega^1 \wedge ... \wedge \omega^n(e_1,...,e_n) = \text{det}(\omega^j(v_i)) = 1. \] Therefore $\omega^1 \wedge ... \wedge \omega^n = 0$ implies that $(\omega^i)$ are dependent. 

\subparagraph*{14.2} {\bf Dimensions in which all covectors are decomposable.}  LOOK IN HARRIS AND COME BACK

\subparagraph*{14.3} {\bf Abstract alternating tensor algebra.} 
\begin{enumerate}
\item is easily verified that if $\omega^i = \omega^j$ for $i \neq j$, then $\text{Alt}(\omega^1 \otimes ... \otimes \omega^n) = 0$. One way of showing this is to break up $\Sigma_n$ into $A_n$ and $(i\, j) \, A_n$ and show cancellation between terms in this sum. Therefore if $f$ is the isomorphism $V^{\ast} \otimes ... \otimes V^{\ast} \simeq T^k(V^{\ast})$, then the composite map $\text{Alt} \circ f$ is well-defined on the quotient $A^k(V^{\ast})$ and so induces a unique homomorphism $F: A^k(V^{\ast}) \rightarrow \Lambda^k(V^{\ast})$. 

To see that $F$ is an isomorphism, we first observe that (in the notation of Proposition 14.8) $f^{-1}(e^I) \in V^{\ast} \otimes ... \otimes V^{\ast}$ and $\text{Alt}(e^I) = e^I$ by Proposition 14.3, so $F ( \pi \circ f^{-1}(e^I)) = e^I$. Therefore the image of $F$ contains a basis for $\Lambda^k(V)$, so $F$ is surjective. 

To see that $F$ is injective, we modify the language of the proof of Proposition 14.8 that elements of the form $\omega^{i_1} \otimes ... \otimes \omega^{i_n} + \mathcal{A}$ with $i_1 < ... < i_n$ span $A^k(V^{\ast})$, where $(\omega_n)$ is a basis for $V^{\ast}$. Injectivity then follows from dimension counting. Let $x = \sum \alpha_I e^{i_1} \otimes ... \otimes e^{i_n}$ be an arbitrary element of $V^{\ast} \otimes ... \otimes V^{\ast}$. Then $x$ is congruent modulo $\mathcal{A}$ to a tensor $x'$ where $\alpha_I$ is $0$ whenever $I$ contains a repeated index. But observe that 
\begin{multline} 
e^{i_1} \otimes e^{i_2} \otimes ... \otimes e^{i_n} = -e^{i_2} \otimes e^{i_1} \otimes ... \otimes e^{i_n} + \\ \bigg( e^{i_1}+e^{i_2} \otimes e^{i_1}+e^{i_2} \otimes ... \otimes e^{i_n} - e^{i_1} \otimes e^{i_1} \otimes ... \otimes e^{i_n}  - e^{i_2} \otimes e^{i_2} \otimes ... \otimes e^{i_n}\bigg) 
\end{multline}
This argument is easily generalized to arbitrary pairs of indices. Therefore permuting indices multiplies a simple tensor by the sign of the permutation. Now tensor products of $V^{\ast}$ act on $V$ in a canonical way. Let $(E_n)$ be the basis dual to $(\omega_n)$. As in the proposition, we therefore have that \[ \sum' \alpha_I e^I(E_{j_1},...,E_{j_n}) =  \sum' \alpha_I \delta^I_J = \alpha_J. \] Therefore $x$ is equal as a function, and therefore equal as a tensor that is a sum of the desired form.  This shows that every element of $V^{\ast} \otimes ... \otimes V^{\ast}$ is congruent modulo $\mathcal{A}$ to a linear combination of tensors of the desired form, so these tensors span the quotient $\mathcal{A}^k(V^{\ast})$. 

\item Let $\mathcal{A}^k$ denote the kernel of $\pi$ when restricted to $(V^{\ast})^{\otimes k}$. The fact that this wedge product is well defined follows from the fact that if $\omega \in \mathcal{A}^k$ and $\eta \in (V^{\ast})^{\otimes \ell}$, then $\omega \otimes \eta \in \mathcal{A}^{k+\ell}$. That $F(\omega \wedge \eta) = \text{Alt}(\tilde{\omega} \otimes \tilde{\eta})$ follows from the commutativity of the diagram.
\end{enumerate}

\subparagraph*{14.5} {\bf Cartan's lemma.} First, we show that the hypothesis implies that each $\alpha^i_p$ is contained in the span of $(\omega^j_p)$ for all $p \in M$. Say that this fails to be true at some fixed $p$ for $\alpha^i_p$. Extend $(\omega^j_p)$ to a basis for $T_pM^{\ast}$. Then $\alpha^i_p = \sum_{j=1}^n c^i_j \omega^j_p$ and $c^i_j \neq 0$ for some $j > k$. Then if $(e_i)$ is the dual basis to $(\omega^i_p)$ \[ \sum_{i=1}^k \alpha^i_p \wedge \omega^i_p(e_j,e_i) = c^i_j \neq 0. \] 

This proves that $\alpha^i_p$ is a linear combination of the $(\omega^j_p)$ for $j = 1,...,k$ and for all $i$. To show that the coefficients are smooth functions of $p$, we can extend $(\omega^j_p)$ to a local frame in a neighborhood of $p$ and simply take the matrix inverse, which is a smooth operation by Cramer's rule. 

\subparagraph*{14.7} {\bf Verifying that $F^{\ast}(d \omega) = d(F^{\ast}\omega)$.} 
\begin{enumerate}
\item \[ F^{\ast}(d\omega) = d(st) \wedge d(e^t) = (t \, ds + s \, dt) \wedge e^t dt = te^t \, ds \wedge dt \]
\[ d(F^{\ast}\omega) = d(st \, e^t \, dt) = te^t \, ds \wedge dt \]
\item \[ F^{\ast}(\omega) =  \sin^2(\theta) \cos(\phi) (\cos(\phi)+2)^2 \, d\theta \wedge d\phi \] 
\[ d\omega = dy \wedge dz \wedge dz \]
\[ d(F^{\ast}\omega) = F^{\ast}(d\omega) = 0 \text{ by dimension} \]
\end{enumerate}

\subparagraph*{14.8} {\bf Naturality of the exterior derivative.} We have already verified that $\Omega^k$ is a contravariant functor, since the pullback map satisfies the necessary properties. Sorting through the definitions (Exercise 11.18), naturality is given by Proposition 14.26. 


\end{document}